\documentclass[11pt,a4paper]{ctexart}
%以下为所使用的宏包
\usepackage{ulem}%下划线
\usepackage{amsmath,amsfonts,amssymb,amsthm,amsbsy}%数学符号
\usepackage{graphicx}%插入图片
\usepackage{booktabs}%三线表
%\usepackage{indentfirst}%首行缩进
\usepackage{tikz}%作图
\usepackage{appendix}%附录
\usepackage{array}%多行公式/数组
\usepackage{makecell}%表格缩并
\usepackage{siunitx}%SI单位--\SI{number}{unit}
\usepackage{mathrsfs}%数学字体
\usepackage{enumitem}%列表间距
\usepackage{multirow}%列表横向合并单元格
\usepackage[colorlinks,linkcolor=red,anchorcolor=blue,citecolor=green]{hyperref}%超链接引用
\usepackage{float}%图片、表格位置排版
\usepackage{pict2e,keyval,fp,diagbox}%带有斜线的表格
\usepackage{fancyvrb,listings}%设置代码插入环境
\usepackage{minted}%代码环境设置
\usepackage{fontspec}%字体设置
\usepackage{color,xcolor}%颜色设置
\usepackage{titlesec} %自定义标题格式
\usepackage{tabularx}%列表扩展
\usepackage{authblk}%titlepage作者信息
\usepackage{nicematrix}%更好的矩阵标定
\usepackage{fbox}%更多浮动体盒子



%以下是页边距设置
\usepackage[left=0.5in,right=0.5in,top=0.81in,bottom=0.8in]{geometry}

%以下是段行设置
\linespread{1.4}%行距
\setlength{\parskip}{0.1\baselineskip}%段距
\setlength{\parindent}{2em}%缩进


%其他设置
\numberwithin{equation}{section}%公式按照章节编号
\newenvironment{point}{\raggedright$\blacktriangleright$}{}
\newenvironment{algorithm}[1]{\vspace{12pt} \hrule\hrule \vspace{3pt} \noindent\textbf{\color[HTML]{E63F00}Algorithm } \,\textit{#1} \vspace{3pt} \hrule\vspace{6pt}}{\vspace{6pt}\hrule\hrule \vspace{12pt}} % 算法伪代码格式环境


%代码环境\lst设置
\definecolor{CodeBlue}{HTML}{268BD2}
\definecolor{CodeBlue2}{HTML}{0000CD}
\definecolor{CodeGreen}{HTML}{2AA1A2}
\definecolor{CodeRed}{HTML}{CB4B16}
\definecolor{CodeYellow}{HTML}{B58900}
\definecolor{CodePurPle}{HTML}{D33682}
\definecolor{CodeGreen2}{HTML}{859900}
\lstset{
    basicstyle=\tt,%字体设置
    numbers=left, %设置行号位置
    numberstyle=\tiny\color{black}, %设置行号大小
    keywordstyle=\color{black}, %设置关键字颜色
    stringstyle=\color{CodeRed}, %设置字符串颜色
    commentstyle=\color{CodeGreen}, %设置注释颜色
    frame=single, %设置边框格式
    escapeinside=`, %逃逸字符(1左面的键),用于显示中文
    %breaklines, %自动折行
    extendedchars=false, %解决代码跨页时,章节标题,页眉等汉字不显示的问题
    xleftmargin=2em,xrightmargin=2em, aboveskip=1em, %设置边距
    tabsize=4, %设置tab空格数
    showspaces=false, %不显示空格
    emph={TRUE,FALSE,NULL,NAN,NA,<-,},emphstyle=\color{CodeBlue2}, %其他高亮}
}


%节标题格式设置
\titleformat{\section}[block]{\large\bfseries}{Exercise \arabic{section}}{1em}{}[]
\titleformat{\subsection}[block]{}{    \arabic{section}.(\alph{subsection})}{1em}{}[]
% \titleformat{\subsubsection}[block]{\normalsize\bfseries}{    \arabic{subsection}-\alph{subsubsection}}{1em}{}[]
% \titleformat{\paragraph}[block]{\small\bfseries}{[\arabic{paragraph}]}{1em}{}[]


% \titleformat{\sectioncommand}[shape]{format}{title-label}{sep}{before-title}[after-title]



% 中文字号
% 初号42pt, 小初36pt, 一号26pt, 小一24pt, 二号22pt, 小二18pt, 三号16pt, 小三15pt, 四号14pt, 小四12pt, 五号10.5pt, 小五9pt


\begin{document}

\begin{center}\thispagestyle{plain}

{\LARGE\textbf{Stat461 - 2023 Fall}}

{\Large\textbf{HW1}}

Tuorui Peng\footnote{TuoruiPeng2028@u.northwestern.edu}
\end{center}

\thispagestyle{myheadings}\markright{Compiled using \LaTeX}
\pagestyle{myheadings}\markright{Tuorui Peng}





  


\section{}
\subsection{}
We use the following equivalent expression for sub-Gaussian and sub-exponential random variables:
\begin{align*}
    \text{Sub-Gaussian: }\exists \theta \geq 0,\,s.t. \mathbb{E}\left[ X^{2k} \right] \leq \dfrac{ (2k)! }{ 2^kk! } \theta^{2k},\, \forall k \in \mathbb{N}^+,\\
    \text{Sub-Exponential: }\mathop{ \sup }\limits_{k\in\mathbb{N}^+} \left[\dfrac{ \mathbb{E}\left[ X^k  \right]  }{ k! } \right]^{1/k} < \infty .
\end{align*}

The proof is as follows:
\begin{itemize}[topsep=2pt,itemsep=0pt]
    \item[$ \Rightarrow $] If $ X $ is sub-Gaussian with such above-mentioned $ \theta $, then we have:
    \begin{align*}
        \left[\dfrac{ \mathbb{E}\left[ (X^2)^k  \right]  }{ k! } \right]^{1/k} \leq & \left[ \dfrac{ (2k)! }{  k!k! } \dfrac{ \theta ^{2k} }{ 2^k }   \right]^{1/k}\\
        =& \left[ \dfrac{ (2k)! }{ (2^kk!)\cdot (2^k k!) }  \right]^{1/k} \cdot 2\theta \\
        =& 2\theta \left[ \prod_{i=1}^k \dfrac{ 2i }{ 2i } \prod_{i=1}^k \dfrac{ 2i-1 }{ 2i }    \right]^{1/k}\\
        \leq & 2\theta  < \infty.
    \end{align*}
    Thus proved that $ X^2 $ is sub-exponential.
    \item[$ \Leftarrow $] If $ X^2 $ is sub-exponential, then take some $ \gamma  $ satisfying the suprema condition, we have:
    \begin{align*}
        \left[\dfrac{ \mathbb{E}\left[ (X^2)^k  \right]  }{ k! } \right]^{1/k} \leq& \gamma \\
         \Rightarrow  \mathbb{E}\left[ X^{2k} \right] \leq & \gamma ^k k! \leq \dfrac{ (2k)! }{ 2^kk! }(\sqrt{\gamma })^{2k},\quad \forall k \in \mathbb{N}^+.
    \end{align*}
    Thus proved that $ X^2 $ is sub-Gaussian with $ \theta \leq  \sqrt{\gamma } $.
\end{itemize}


\subsection{}
We use the following equivalent expression for sub-Gaussian and sub-exponential random variables:\footnote{
    which are straightforward to prove using the definition of sub-Gaussian and sub-exponential random variables. Taking sub-gaussian as an example, we have:
    \begin{align*}
        \mathbb{E}\left[ \left\vert X \right\vert ^p \right] =& \int_{0}^\infty \mathbb{P}\left( \left\vert X \right\vert ^p > t \right) \,\mathrm{d}t\\
        =& \int_{0}^\infty \mathbb{P}\left( \left\vert X \right\vert > t \right) pt^{p-1} \,\mathrm{d}t\\
        \leq & \int_{0}^\infty 2e^{-t^2/2\sigma ^2}pt^{p-1} \,\mathrm{d}t\\
        =& p \Gamma (p/2) \sigma ^p 2^{p/2}\\
        \leq & p (\sqrt{p}  \sigma )^p 
    \end{align*}
    
    
}
\begin{align*}
    \text{Sub-Gaussian: }\exists K_2 >0,\, s.t. \forall p\geq 1,\, \mathbb{E}\left[ \left\vert X-\mathbb{E}\left[ X \right]  \right\vert ^p  \right] ^{1/p} \leq K_2\sqrt{p}\\
    \text{Sub-Exponential: }\exists K_1 >0,\, s.t. \forall p\geq 1,\, \mathbb{E}\left[ \left\vert X-\mathbb{E}\left[ X \right]  \right\vert ^p  \right] ^{1/p} \leq K_1p
\end{align*}

Using the equivalentce, the proof is straightforward. Assume the $ K_2 $ parameter for $ X $, $ Y $ are $ K_{2,X} $ and $ K_{2,Y} $ respectively and WLOG assume $ X,Y $ are both mean-zero. Using the convexity of $ \left\Vert \exp \, \cdot \,  \right\Vert _p :=\mathbb{E}\left[ \left\vert e^{\, \cdot \, }  \right\vert ^p  \right] ^{1/p}  $, we have:
\begin{align*}
    \mathbb{E}\left[ \left\vert XY \right\vert^p  \right]^{1/p}= &  \left\Vert   e^{\frac{ \log \left\vert X^2  \right\vert + \log \left\vert Y^2 \right\vert }{ 2 } } \right\Vert _p \\
    \leq & \dfrac{ \left\Vert e^{\log \left\vert X^2  \right\vert } \right\Vert _p + \left\Vert e^{\log \left\vert Y^2  \right\vert } \right\Vert _p }{ 2 }\\
    =& \dfrac{ 1 }{ 2 } \mathbb{E}\left[ \left\vert X \right\vert ^{2p}  \right] ^{1/p} + \dfrac{ 1 }{ 2 } \mathbb{E}\left[ \left\vert Y \right\vert ^{2p}  \right] ^{1/p}\\
    \leq & \dfrac{ 1 }{ 2 }( K_{2,X}\sqrt{p})^2 + \dfrac{ 1 }{ 2 }( K_{2,Y}\sqrt{p})^2\\
    =& \dfrac{ K_{2,X}^2 + K_{2,Y}^2 }{ 2 } p ,\qquad \forall p\geq 1
\end{align*}



\section{}
Denote the polynomial side bound as 
\begin{align*}
    M:= \mathop{ \inf  }\limits_{k\in \mathbb{N}} \dfrac{ \mathbb{E}\left[ X^k  \right]  }{ \delta ^k }   
\end{align*}

Using the taylor expansion of $ x\mapsto e^x $ at $ 0 $, we have
\begin{align*}
     \mathbb{E}\left[ e^\lambda X \right] =&  \sum_{k=0}^\infty \dfrac{ \lambda ^k }{ k! } \mathbb{E}\left[ X^k \right] \\
     \geq &  \sum_{k=0}^\infty \dfrac{ \lambda ^k }{ k! } \delta ^k M\\
     =& Me^{\lambda \delta } 
\end{align*}
Thus we have: 
\begin{align*}
    \dfrac{\mathbb{E}\left[ e^\lambda X \right]  }{ e^{\lambda \delta }  }  \geq M = \mathop{ \inf  }\limits_{k\in \mathbb{N}} \dfrac{ \mathbb{E}\left[ X^k  \right]  }{ \delta ^k }    ,\quad \forall \lambda >0
\end{align*}
thus proved the result.


\section{}

Consider a random variable $ X\sim \mathrm{Binom}(n,1/2) $, which has Moment Generating Function $ M_X(\lambda ) = (e^\lambda /2 + 1/2)^n $. Using Chernoff bound we have:
\begin{align*}
    \dfrac{ 1 }{ 2^n }\sum_{j=1}^k \binom{n}{j} = & \mathbb{P}\left( X \geq n-k \right) \\
    \leq & \inf _{\lambda >0} \big\{ n \log \dfrac{ e^\lambda +1 }{ 2 } - \lambda (n -k)  \big\} := f(\lambda )
\end{align*}
we optimize the bound w.r.t. $ \lambda $. $  \mathrm{R.H.S.} $ takes minimum at $ \lambda =\log \dfrac{ n-k }{ k }  $, yielding the upper bound:
\begin{align*}
    \log \sum_{j=1}^k \binom{n}{j}- n\log 2\leq & -n\log 2+ n\log \dfrac{ n }{ k } -(n-k)\log \dfrac{ n-k }{ k }
\end{align*}

Now it suffces to prove the following inequality:
\begin{align*}
    n\log \dfrac{ n }{ k } -(n-k)\log \dfrac{ n-k }{ k } \leq k\log \dfrac{ ne }{ k } 
\end{align*}
taking $ \alpha = \dfrac{ k }{ n } \in (0,1] $, it suffices to verify the following inequality:
\begin{align*}
    f(\alpha ):= \log \dfrac{ 1 }{ 1-\alpha   } + \alpha \log \dfrac{ 1-\alpha  }{ \alpha  } -\alpha \log \dfrac{ e }{ \alpha  } \leq 0  
\end{align*}
which can be easily verified that $ f'(\alpha )\leq 0 $, and notice that $ f(0)=0 $, thus proved the inequality. And we have
\begin{align*}
    \sum_{j=1}^k \binom{n}{j} \leq \left(\dfrac{ en  }{ k } \right)^k
\end{align*}




\section{}

\subsection{}
We have
\begin{align*}
    var(X)= & \mathbb{E}\left[ (X-\mathbb{E}\left[ X \right] )^2 \right]  \\
    =& \int_{0}^\infty \mathbb{P}\left( (X-\mathbb{E}\left[ X \right] )^2 > t \right) \,\mathrm{d}t \\
    \leq & \int_{0}^\infty c_1e^{-c_2t} \,\mathrm{d}t \\
    =& \dfrac{ c_1 }{ c_2 }
\end{align*}

\subsection{}

We can use example that $ X=1,2,3,4 $ equiprobably, for which median can take any number in $ [2,3] $

\subsection{}

Using Chebyshev's inequality, we have:
\begin{align*}
    \mathbb{P}\left( \left\vert X- \mu  \right\vert \geq t  \right)   \leq \dfrac{ \sigma ^2 }{ t^2 }
\end{align*}
taking $ t=\sqrt{2}\sigma  $ we can obtain that $ \left\vert m_X-\mu  \right\vert \leq \sqrt{2 }\sigma  $. Using the fact we have:
\begin{itemize}[topsep=2pt,itemsep=0pt]
    \item For $ t \geq \sqrt{2}\sigma  $:
    \begin{align*}
        \mathbb{P}\left( \left\vert X-m_X \right\vert  \geq t \right) \leq & \mathbb{P}\left( \left\vert X-\mu  \right\vert + \left\vert \mu -m_X \right\vert \geq t \right)\\
        =& \mathbb{P}\left( \left\vert X-\mu  \right\vert \geq t-\left\vert \mu -m_X \right\vert  \right)\\
        \leq &c_1e^{-c_2(t-\sqrt{2}\sigma )^2} \leq \tilde{c}_3e^{-c_4t^2}
    \end{align*}
    \item For $ 0 < t < \sqrt{2}\sigma  $: 
    \begin{align*}
        \mathbb{P}\left( \left\vert X-m_X \right\vert  \geq t \right)  \leq 1
    \end{align*}
\end{itemize}
To combine the two case, we just need to consider resetting $ \tilde{c}_3 $ s.t.
\begin{align*}
    \tilde{c}_3e^{-c_4^{>}t^2}\Big|_{t=\sqrt{2}\sigma } = \tilde{c_3}e^{-2\sigma ^2c_4} \geq 1
\end{align*}
i.e. we can take
\begin{align*}
    c_3 = \max \{ \tilde{c}_3, e^{2\sigma ^2c_4} \} ,\quad c_4 = c_2
\end{align*}



\subsection{}
To prove the converse statement, the idea is exactly the same by using $ \left\vert \mu -m_X \right\vert \leq \sqrt{2}\sigma  $:
\begin{itemize}[topsep=2pt,itemsep=0pt]
    \item For $ t \geq \sqrt{2}\sigma  $:
    \begin{align*}
        \mathbb{P}\left( \left\vert X-\mu  \right\vert  \geq t \right) \leq & \mathbb{P}\left( \left\vert X-m_X  \right\vert + \left\vert \mu -m_X \right\vert \geq t \right)\\
        =& \mathbb{P}\left( \left\vert X-m_X  \right\vert \geq t-\left\vert \mu -m_X \right\vert  \right)\\
        \leq &c_3e^{-c_4(t-\sqrt{2}\sigma )^2} \leq \tilde{c_1} e^{-{c_2}t^2}
    \end{align*}
    \item For $ 0 < t < \sqrt{2}\sigma  $: 
    \begin{align*}
        \mathbb{P}\left( \left\vert X-m_X \right\vert  \geq t \right)  \leq 1
    \end{align*}
\end{itemize}
To combine the two case, we just need to consider resetting $ \tilde{c}_1 $ s.t.
\begin{align*}
    \tilde{c}_1e^{-c_2^{>}t^2}\Big|_{t=\sqrt{2}\sigma } = \tilde{c_1}e^{-2\sigma ^2c_2} \geq 1
\end{align*}
i.e. we can take
\begin{align*}
    c_1 = \max \{ \tilde{c}_1, e^{2\sigma ^2c_2} \} ,\quad c_2 = c_4 
\end{align*}

\textbf{Note:} A better parameter $ c_{\, \cdot \, } $ is possible if we treat the "boundary" between the two cases more carefully.

\section{}


\subsection{}

Denote $ \tilde{X}_i=X_i/\sqrt{2}\sigma  $, which should satisfy tail bound 
\begin{align*}
    \mathbb{P}\left( \left\vert \tilde{X}_i \right\vert \geq t \right)  \leq 2e^{-t^2}
\end{align*}
we have $ \forall q\geq 1 $:
\begin{align*}
     \mathbb{E}\left[ \left\vert \tilde{X}_i \right\vert ^q \right] =& \int_{0}^\infty \mathbb{P}\left( \left\vert \tilde{X}_i \right\vert ^q > t \right) \,\mathrm{d}t\\
    =& \int_{0}^\infty \mathbb{P}\left( \left\vert \tilde{X}_i \right\vert > t \right) qt^{q-1} \,\mathrm{d}t\\
    \leq & \int_{0}^\infty 2e^{-t^2}qt^{q-1} \,\mathrm{d}t\\
    =& q \Gamma (q/2) \leq q(\dfrac{ q }{ 2 } )^{q/2}
\end{align*}
thus we have
\begin{align*}
    \mathbb{E}\left[ \left\vert X_i \right\vert ^q \right] \leq (\sqrt{2}\sigma )^q q(\dfrac{ q }{ 2 } )^{q/2} = q (\sigma^2 q)^{q/2}
\end{align*}



Then using the concavity of $ x\mapsto x^{1/q} $, we have:
\begin{align*}
    \mathbb{E}\left[ \left\Vert X \right\Vert _q \right]  \leq & \bigl(\mathbb{E}\left[ \sum_{i=1}^n \left\vert X_i \right\vert ^q \right] \bigr)^{1/q}\\
    =& \bigl(\sum_{i=1}^n \mathbb{E}\left[ \left\vert X_i \right\vert ^q \right] \bigr)^{1/q}\\
    \leq & \bigl(nq (\sigma^2 q)^{q/2} \bigr)^{1/q}\\
    =& (nq)^{1/q} (\sigma^2 q)^{1/2}\\
    =& q^{1/q} n^{1/q} \sigma \sqrt{q}
\end{align*}
Further note that $ x\mapsto x^{1/x} $ has maximum at $ x=e $ of maximum value $ 1.44\dots <4 $, we have:
\begin{align*}
    \mathbb{E}\left[ \left\Vert X \right\Vert _q \right]  \leq 4n^{1/q} \sigma \sqrt{q}
\end{align*}

\subsection{}
Note that we have relation:
\begin{align*}
    \max \left\vert X_i  \right\vert \leq \left\Vert X_i  \right\Vert _q,\quad \forall q \geq 1 
\end{align*}
we only need to optimize the above bound for $ \mathbb{E}\left[ \left\Vert X \right\Vert _q \right] \leq 4n^{1/n}\sigma \sqrt{q} $, which takes minimum at $ q=2\log n $, yielding a bound of 
\begin{align*}
    \mathbb{E}\left[ \max \left\vert X_i  \right\vert \right] \leq& \mathop{ \inf  }\limits_{q\geq 1} \mathbb{E}\left[ \left\Vert X \right\Vert _q \right] \\
    \leq & \mathop{ \inf  }\limits_{q\geq 1} 4n^{1/q} \sigma \sqrt{q} \\
    =& 4\sigma \sqrt{2e\log n}\\
    \leq & 4e\sigma \sqrt{\log n}
\end{align*}







































































\end{document}