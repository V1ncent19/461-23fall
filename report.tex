\documentclass[11pt,a4paper]{article}
%以下为所使用的宏包
\usepackage{ulem}%下划线
\usepackage{amsmath,amsfonts,amssymb,amsthm,amsbsy}%数学符号
\usepackage{graphicx}%插入图片
\usepackage{booktabs}%三线表
%\usepackage{indentfirst}%首行缩进
\usepackage{tikz}%作图
\usepackage{appendix}%附录
\usepackage{array}%多行公式/数组
\usepackage{makecell}%表格缩并
\usepackage{siunitx}%SI单位--\SI{number}{unit}
\usepackage{mathrsfs}%数学字体
\usepackage{enumitem}%列表间距
\usepackage{multirow}%列表横向合并单元格
\usepackage[colorlinks,linkcolor=red,anchorcolor=blue,citecolor=red]{hyperref}%超链接引用
\usepackage{float}%图片、表格位置排版
\usepackage{pict2e,keyval,fp,diagbox}%带有斜线的表格
\usepackage{fancyvrb,listings}%设置代码插入环境
\usepackage{minted}%代码环境设置
\usepackage{fontspec}%字体设置
\usepackage{color,xcolor}%颜色设置
\usepackage{titlesec} %自定义标题格式
\usepackage{tabularx}%列表扩展
\usepackage{authblk}%titlepage作者信息
\usepackage{nicematrix}%更好的矩阵标定
\usepackage{fbox}%更多浮动体盒子



%以下是页边距设置
\usepackage[left=0.5in,right=0.5in,top=0.81in,bottom=0.8in]{geometry}

%以下是段行设置
\linespread{1.4}%行距
\setlength{\parskip}{0.1\baselineskip}%段距
\setlength{\parindent}{2em}%缩进


%其他设置
\numberwithin{equation}{section}%公式按照章节编号
\newenvironment{point}{\raggedright$\blacktriangleright$}{}
\newenvironment{algorithm}[1]{\vspace{12pt} \hrule\hrule \vspace{3pt} \noindent\textbf{\color[HTML]{E63F00}Algorithm } \,\textit{#1} \vspace{3pt} \hrule\vspace{6pt}}{\vspace{6pt}\hrule\hrule \vspace{12pt}} % 算法伪代码格式环境


%代码环境\lst设置
\definecolor{CodeBlue}{HTML}{268BD2}
\definecolor{CodeBlue2}{HTML}{0000CD}
\definecolor{CodeGreen}{HTML}{2AA1A2}
\definecolor{CodeRed}{HTML}{CB4B16}
\definecolor{CodeYellow}{HTML}{B58900}
\definecolor{CodePurPle}{HTML}{D33682}
\definecolor{CodeGreen2}{HTML}{859900}
\lstset{
    basicstyle=\tt,%字体设置
    numbers=left, %设置行号位置
    numberstyle=\tiny\color{black}, %设置行号大小
    keywordstyle=\color{black}, %设置关键字颜色
    stringstyle=\color{CodeRed}, %设置字符串颜色
    commentstyle=\color{CodeGreen}, %设置注释颜色
    frame=single, %设置边框格式
    escapeinside=`, %逃逸字符(1左面的键),用于显示中文
    %breaklines, %自动折行
    extendedchars=false, %解决代码跨页时,章节标题,页眉等汉字不显示的问题
    xleftmargin=2em,xrightmargin=2em, aboveskip=1em, %设置边距
    tabsize=4, %设置tab空格数
    showspaces=false, %不显示空格
    emph={TRUE,FALSE,NULL,NAN,NA,<-,},emphstyle=\color{CodeBlue2}, %其他高亮}
}


%节标题格式设置
% \titleformat{\section}[block]{\large\bfseries}{Exercise \arabic{section}}{1em}{}[]
% \titleformat{\subsection}[block]{}{    \arabic{section}.(\alph{subsection})}{1em}{}[]
% \titleformat{\subsubsection}[block]{\normalsize\bfseries}{    \arabic{subsection}-\alph{subsubsection}}{1em}{}[]
% \titleformat{\paragraph}[block]{\small\bfseries}{[\arabic{paragraph}]}{1em}{}[]


% \titleformat{\sectioncommand}[shape]{format}{title-label}{sep}{before-title}[after-title]



% 中文字号
% 初号42pt, 小初36pt, 一号26pt, 小一24pt, 二号22pt, 小二18pt, 三号16pt, 小三15pt, 四号14pt, 小四12pt, 五号10.5pt, 小五9pt


\begin{document}

\begin{center}\thispagestyle{plain}

{\LARGE\textbf{Stat450 2025 Winter - Final Report}}

{\Large\textbf{On Minimax Rate of $ \ell_1 $ norm in Gaussian Location Model}}

Tuorui Peng\footnote{TuoruiPeng2028@u.northwestern.edu}
\end{center}

\thispagestyle{myheadings}\markright{Compiled using \LaTeX}
\pagestyle{myheadings}\markright{Tuorui Peng}


\begin{abstract}
    In this project, we followed the idea of \cite{cai2011testing} to present the minimax rate for estimating non-smooth functions of mean in Gaussian location model. Specifically, we focused on the estimation of $ \left\Vert \theta  \right\Vert _1 := \sum_{i=1}^p \left\vert \theta _i \right\vert  $ from an observation $ X\sim \mathcal{N}(\theta ,I_p) $. The main result presented is that: the minimax rate under $ \ell_2 $ loss is $ M^2p^2\left(\frac{\log\log p}{\log p}\right)^2 $ for $ M $-bounded parameter, and $ \frac{p^2}{\log p} $ for unbounded case.
\end{abstract}


\section{Motivation and Introduction}

\subsection{Introduction}

Gaussian location model is a basic and important model in statistics. Here we focus on the $ \ell_2^2 $ loss Minimax rate for location parameter estimation. To be specific, we have $ X_1,\ldots, X_n\mathop{ \sim  }\limits^{i.i.d.}   \mathcal{N}(\theta , I_p )$ where $ \theta \in \Theta _p\subseteq \mathbb{R}^p $ is the parameter of interest. Some estimation problem includes $ T(\theta )=\theta  $, $ T(\theta )=\theta _{\max} $, $ T(\theta )= \left\Vert \theta  \right\Vert _2 $, $ T(\theta )=\left\Vert \theta  \right\Vert  _1 $, etc. The minimax rate under $ \ell_2^2 $ loss for these problems are well studied.
\begin{align*}
    \mathfrak{M}^*(T,\Theta _p):=&\mathop{ \inf }\limits_{\hat{T}} \mathop{ \sup }\limits_{\theta \in \Theta _p} \mathbb{E}_{  }\left[ (T(\theta) -\hat{T}(Y))^2 \right] 
\end{align*}
For example, we have some well-known results as follows:
\begin{table}[H]
    \centering
    \renewcommand\arraystretch{1.15}
    \begin{tabular}{|c|c|}
        \hline
        Estimation Problem & Minimax Rate \\
        \hline
        $ T(\theta )=\theta  $ & $ \frac{p}{n} $ \\
        \hline
        $ T(\theta )=\theta _{\max}  $ & $ \frac{p^2}{n} $ \\
        \hline
        $ T(\theta )=\left\Vert \theta  \right\Vert _2 $ & $ \frac{\sqrt{p}}{n} $ \\
        \hline
    \end{tabular}
\end{table}
In the lecture, we already proved the case of $ T(\theta )=\theta  $ (c.f. Lecture 5) and $ T(\theta )=\left\Vert \theta  \right\Vert _2 $ (c.f. Lecture 7\footnote{We got that $ \varepsilon ^* \asymp d^{1/4}/n^{1/2} $. Applying Le Cam's two-point method we have $ \mathfrak{M}^* \gtrsim (\varepsilon ^*)^2 \asymp d/n $.}). Here in this project, we will focus on the case of $ T(\theta )=\left\Vert \theta  \right\Vert  _1 $.

Due to the property of $ \ell_2^2 $ loss, it suffices to consider one sample case and deduce that $ \mathfrak{M}_{n=n}=\mathfrak{M}_{n=1}/n $. i.e. our setting is that $ X\sim \mathcal{N}(\theta ,I_p) $ and the task is to estimate $ \left\Vert \theta  \right\Vert  _1 :=\sum_{i=1}^p \left\vert \theta _i \right\vert  $.


\subsection{Challenge}

$ \left\Vert \theta  \right\Vert _1 $ is a typical example of a non-smooth functional of parameter. While the distribution of normal distribution is `much too nice' by being analytic. e.g. our first trial of estimation might be $ \hat{T}(X)=\left\Vert X \right\Vert _1  $, however we would immediately see that it's biased by noticing that (take 1-dim as example):
\begin{align*}
    \int _\mathbb{R} \left\vert X \right\vert \dfrac{ 1 }{ \sqrt{2\pi} }\exp\left[ -\dfrac{ (x-\theta ) ^2}{ 2 }  \right]  \,\mathrm{d}x \mathop{ = }\limits^{?}\left\vert \theta  \right\vert   
\end{align*}
in which left hand side is a smooth function of $ \theta  $ and right hand side is not. Actually no estimator $ \hat{T}(X) $ can satisfy the analytic property, so we have no access to an unbiased estimator to $ \left\vert \theta  \right\vert  $.

However, we do have access to some other analytic estimators. For example, we can safely estimate moments (i.e. polynomial functions) of $ \theta $. Thus here the key recipe is to approximate $ \left\vert \theta  \right\vert  $ by a polynomial function (of proper degree) of $ \theta  $.



\subsection{Outline of the Proof}
We would first do the proof for bounded case:
\begin{enumerate}[topsep=2pt,itemsep=2pt]
    \item We can approximate $ \left\vert x \right\vert  $ by a degree-$ d $ polynomial up to $ \asymp \frac{1}{d} $ precision;
    \item The variance of the polynomial estimator is $ \asymp p\cdot d! $;
    \item A bias-variance trade-off gives us a desired upper bound side;
    \item Control $ \chi^2 $ divergence to get the lower bound side.
\end{enumerate}
Then a extension to unbounded case is easy by noticing that: when $ \left\Vert \theta  \right\Vert _1 $ is large enough, $ \left\Vert X  \right\Vert   $ would be similar to $ \left\Vert \theta  \right\Vert _1 $, so we only need to consider the case that $ \left\Vert \theta  \right\Vert _1 $ is small enough, and study the rate of the `boundary', then go back to bounded case.



\section{Useful Knowledge}

\subsection{Polynomial Approximation of $ \left\vert x  \right\vert  $}\label{sec:polyapprox}

A main recipe is the approximation of $ \left\vert x \right\vert , x\in [-1,1] $ by a polynomial function. In this project, we use $ d $ for the degree of polynomial, and we seek a best degree-$ d $ polynomial estimator $ G_d^*(x) $ in the sense that reaches:
\begin{align*}
    \delta _d(f=\left\vert \, \cdot \,  \right\vert )=\mathop{ \inf }\limits_{G_d\in \mathcal{P}_d} \mathop{ \max }\limits_{x\in [-1,1]} \big\vert \left\vert x \right\vert - G_d(x) \big\vert 
\end{align*}
i.e. optimal in $ \left\Vert \, \cdot \,  \right\Vert _\infty $, in which $ \mathcal{P}_d $ is the set of degree-$ d $ polynomial functions. The classical Chebyshev's alternation theorem \cite{cai2011testing} states that the optimal polynomial $ G_d^*(x) $ is s.t.: $ \left\vert x \right\vert - G_d^*(x) $ take value consecutively $ \delta _d,-\delta _d, \delta _d,\ldots $. And Berstein \cite{bernstein1914meilleure} proved that this optimal error is of order $ \asymp \frac{1}{d} $:
\begin{align*}
    \lim_{d\to \infty }d\delta _d(f=\left\vert \, \cdot \,  \right\vert )=\beta _*\approx 0.2802\tag{Berstein's constant}
\end{align*}
Upgrade to $ p $ dimensional, we have that the optimal polynomial approximation of $\left\Vert \theta  \right\Vert _1 $ with precision rate $ \asymp \frac{p}{d} $.



\subsection{Hermite Polynomial}

Hermite polynomial(s) is a set of orthogonal polynomial with respect to the normal density weight function $ \exp\left[ -\frac{ x^2}{ 2 }  \right]  $, defined as:
\begin{align*}
    H_r(x) :=(-1)^r\exp\left[ \frac{ x^2}{ 2 }  \right] \dfrac{ \mathrm{d}^r }{ \mathrm{d}x^r }\exp\left[ -\frac{ x^2}{ 2 }  \right]
\end{align*}
with orthogonality:
\begin{align*}
    \left\langle H_r, H_s \right\rangle =&\int _\mathbb{R} H_r(x)H_s(x)\exp\left[ -\frac{ x^2}{ 2 }  \right] \,\mathrm{d}x = \mathbb{E}_{ Z\sim \mathcal{N}(0,1) }\left[ H_r(Z)H_s(Z) \right]  = r!\delta _{rs}
\end{align*}
and it form a direct estimator to moments:
\begin{align*}
    \mathbb{E}_{ Z\sim \mathcal{N}(\theta ,1) }\left[ H_r(Z) \right] =&\int _\mathbb{R} H_r(x)\exp\left[ -\frac{ (x-\theta)^2}{ 2 }  \right] \,\mathrm{d}x = \theta ^r
\end{align*}

When using Hermite polynomial up to degree $ d $ to approximate $ \left\vert x \right\vert _1  $ as
\begin{align*}
    \left\vert x \right\vert _1 \approx G^*_d(x) = \sum_{r=0}^d \alpha _rH_r(x),\quad x\in [-1,1],
\end{align*}
we have that $ \alpha _r  \lesssim \mathrm{const}^{\mathrm{const} \cdot d} $, i.e. $ \log \alpha _r \lesssim d $ \cite{cai2011testing}.\footnote{Intuitively, $ \alpha _rx^r \lesssim 1 $.}








\section{Bounded Case}

We first focus on the bounded case that $ \theta \in \Theta _p(M):= \{\theta \in \mathbb{R}^p: \left\vert \theta  \right\vert _\infty \leq M \} $ for some $ M>0 $. WLOG we assume $ M=1 $.




\subsection{Upper Bound Side}

We use the above mentioned polynomial approximation (the degree to be determined later) to construct a (good enough) estimator, which would give us a upper bound side of the minimax rate.
\begin{align*}
    \hat{T}_i(X)= G^*_d(X_i) = \sum_{r=0}^d \alpha _rH_r(X_i),\quad i \in [p],
\end{align*}
for which we know that the $ \ell_2^2 $ loss can be decomposed to the following bias-variance trade-off:
\begin{align*}
     \text{Bias: }&\left\vert \hat{T}(X)- \left\vert X \right\vert _1  \right\vert = \delta _d(f=\left\vert \, \cdot \,  \right\vert ) \lesssim \dfrac{ p }{ d },\\
     \text{Variance: }&var(\hat{T}(X))=\sum_{i=1}^p var(\hat{T}_i(X)) \lesssim p\cdot d! \cdot  \mathrm{const}^{\mathrm{const} \cdot d} \lesssim p\cdot d!.
\end{align*}
To obtain the optimal loss rate, we have:
\begin{align*}
    \text{optimal }d: &\quad (\dfrac{ p }{ d } )^2 \asymp p\cdot d! \mathop{  \Rightarrow  }\limits^{(i)}  d_p\asymp \dfrac{ \log p }{ \log\log p } \\
     \Rightarrow \text{optimal loss: }&\quad \mathfrak{M}^* \lesssim \left( \dfrac{ p }{ d_p } \right) ^2 +p\cdot d_p! \lesssim p^2\left( \dfrac{ \log\log p }{ \log p } \right) ^2.
\end{align*}
where $(i)$ uses Stirling's formula $ d! \sim (d/e)^d $.



\subsection{Lower Bound Side}

We use the $ \chi^2 $ divergence + Le Cam's two-point method we introduced in the lecture to get the lower bound side. We wanna construct 




\begin{point}
    (Upper) Bounding $ \chi^2 $ divergence by selecting proper prior and degree of polynomial.
\end{point}

Denote the (posterior) distribution of Null and Alternative as $ f_0(x)=\pi_0(\theta ) \otimes \phi(x;\theta ) = \int_\mathbb{R} \phi (x-t) \pi_0(\,\mathrm{d}t) $ and $ f_1(x)=\pi_1(\theta ) \otimes \phi(x;\theta ) = \int_\mathbb{R} \phi (x-t) \pi_1(\,\mathrm{d}t) $ respectively. We have the following two useful lemmas:
\begin{itemize}[topsep=2pt,itemsep=0pt]
    \item By the convexity of $ \xi \mapsto e^{-\xi } $, we have
    \begin{align*}
        f_0(x) =&\int_\mathbb{R} \phi (x-t) \pi_0(\,\mathrm{d}t) \\
        \geq&\dfrac{ 1 }{ \sqrt{2\pi} }\exp\left[ -\int_\mathbb{R} \dfrac{ (x-t)^2 }{ 2 }\pi_0(\,\mathrm{d}t)  \right] \\
        =& \phi (x)\exp\left[ -\dfrac{ 1 }{ 2 } \int t^2 \pi_0(\,\mathrm{d}t)  \right]\\
        \geq&\phi (x)\exp\left[ -\dfrac{ 1 }{ 2 } \right]
    \end{align*}
    \item Taylor expansion of normal distribution:
    \begin{align*}
        \phi (x-t)=&\sum_{k=0}^\infty H_k(x)\phi (x) \dfrac{ t^k }{ k! }.  
    \end{align*}
    and thus we have
    \begin{align*}
        f_1(x)=&\int_\mathbb{R} \phi (x-t) \pi_1(\,\mathrm{d}t) \\
        =&\int_\mathbb{R} \sum_{k=0}^\infty H_k(x)\phi (x) \dfrac{ t^k }{ k! } \pi_1(\,\mathrm{d}t) \\
        =&\sum_{k=0}^\infty \dfrac{ 1 }{ k! }  H_k(x)\phi (x)\int_\mathbb{R}  t^k \pi_1(\,\mathrm{d}t)
    \end{align*}
    (and similarly for $ f_0(x) $).
    \item We study the following $ \chi^2 $ divergence:
    \begin{align*}
        \chi^2(f_1\big\Vert f_0) =&\int_\mathbb{R} \dfrac{ (f_1(x)-f_0(x))^2 }{ f_0(x) }\,\mathrm{d}x\\
        \mathop{ = }\limits^{(i)} &\int_\mathbb{R} \dfrac{ \big( \sum_{k=0}^\infty \frac{ 1 }{ k! }  H_k(x)\phi (x) \int_{-1}^1  t^k (\pi_1(\,\mathrm{d}t) - \pi_0(\,\mathrm{d}t) )  \big)^2 }{ f_0(x) }\,\mathrm{d}x\\ 
        \mathop{ = }\limits^{(ii)}&\int_\mathbb{R} \dfrac{ \big( \sum_{k=d+1}^\infty \frac{ 1 }{ k! }  H_k(x)\phi (x) \int_{-1}^1  t^k (\pi_1(\,\mathrm{d}t) - \pi_0(\,\mathrm{d}t) )  \big)^2 }{ f_0(x) }\,\mathrm{d}x\\ 
        \mathop{ \leq }\limits^{(iii)}&e^{1/2}\int_\mathbb{R}  \big( \sum_{k=d+1}^\infty \frac{ 1 }{ k! }  H_k(x)\phi (x) \int_\mathbb{R}  t^k (\pi_1(\,\mathrm{d}t) - \pi_0(\,\mathrm{d}t) )  \big)^2\phi (x)^{-1} \,\mathrm{d}x\\
        \mathop{ = }\limits^{(iv)}&e^{1/2}\sum_{k=d+1}^\infty \dfrac{ 1 }{ k! } \big(\int_{-1}^1t^k \pi_1(\,\mathrm{d}t) - \int_{-1}^1t^k \pi_0(\,\mathrm{d}t) \big)^2 \\
        \leq&e^{1/2}\sum_{k=d+1}^\infty \dfrac{ 1 }{ k! } \\
        \mathop{ \lesssim }\limits^{(v)}& \dfrac{ 1 }{ d! }
    \end{align*}
    where $ (i) $ using the above mentioned Taylor expansion, $ (ii) $ since our prior is selected s.t. having matching first $ d $ moments, $ (iii) $ using the lemma mentioned above, and $ (iv) $ using the orthogonality of Hermite polynomial, and $ (v) $ since $ 1/k! $ decays faster than geometric series so is also dominated by the leading them $ 1/d! $.
\end{itemize}
Thus we have:
\begin{align*}
    \chi^2\left(\mathbb{E}_{ \theta \sim \pi_1^{\otimes p} }\left[ p_\theta ^{\otimes p} \right] \big\Vert \mathbb{E}_{ \theta \sim \pi_0^{\otimes p} }\left[ p_\theta ^{\otimes p} \right] \right) +1 =& \chi^2\left( f_1 ^{\otimes p} \big\Vert f_0 ^{\otimes p} \right)\\
    \mathop{ = }\limits^{(i)}& \Big( \chi^2(f_1\big\Vert f_0)  +1 \Big) ^p\\
    \leq& (\dfrac{ c }{ d! } +1 )^p\\
    \leq& \exp\left[ \dfrac{ cp }{ d! }  \right]
\end{align*}
Thus by selecting $ d_p\gtrsim  \dfrac{ \log p }{ \log\log p } $ we have $ \chi^2\left(\mathbb{E}_{ \theta \sim \pi_1^{\otimes p} }\left[ p_\theta ^{\otimes p} \right] \big\Vert \mathbb{E}_{ \theta \sim \pi_0^{\otimes p} }\left[ p_\theta ^{\otimes p} \right] \right)  \lesssim 1 $ and thus by Le Cam's two-point method we have:
\begin{align*}
    \mathfrak{M}^* \gtrsim & \varepsilon ^2\big(1 - d_\mathrm{ TV }\left(\mathbb{E}_{ \theta \sim \pi_1^{\otimes p} }\left[ p_\theta ^{\otimes p} \right] \big\Vert \mathbb{E}_{ \theta \sim \pi_0^{\otimes p} }\left[ p_\theta ^{\otimes p} \right] \right)\big) \\
    \geq& \varepsilon ^2\big(1 - \dfrac{ 1 }{ 2 }\sqrt{\chi^2 \left(\mathbb{E}_{ \theta \sim \pi_1^{\otimes p} }\left[ p_\theta ^{\otimes p} \right] \big\Vert \mathbb{E}_{ \theta \sim \pi_0^{\otimes p} }\left[ p_\theta ^{\otimes p} \right] \right) } \big)\\
    \gtrsim& \varepsilon ^2
\end{align*}
where for the rate of $ \varepsilon  $, we first describe its intuition, then formalize it. 


\begin{point}
    (Lower) Bounding the gap between Null and Alternative.
\end{point}

\textbf{Intuition: }As we mentioned in \autoref{sec:polyapprox}, a polynomial of degree $ d $ can approximate $ \left\vert x \right\vert  $ up to precision $ \delta _d(\left\vert \, \cdot \,  \right\vert ) = \frac{1}{d} $, so the smallest gap between Null and Alternative should also be of order $ \asymp \frac{1}{d} $.

\textbf{Formalization: }We can formalize the above as a optimization problem:
\begin{align*}
    \mathop{ \arg\max }\limits_{\pi_0,\pi_1}& \mathbb{E}_{ \pi_0 }\left[ \left\vert \theta  \right\vert  \right] -\mathbb{E}_{ \pi_1 }\left[ \left\vert \theta  \right\vert  \right]\\
    w.r.t.& \mathbb{E}_{ \pi_0 }\left[ \theta ^k \right] =\mathbb{E}_{ \pi_1 }\left[ \theta ^k \right], k=0,1,\ldots,d
\end{align*}
i.e. we just take $ \pi_0,\pi_1 $ as the optimizer and it would yield $ \varepsilon =  \mathbb{E}_{ \pi_0 }\left[ \left\vert \theta  \right\vert  \right] -\mathbb{E}_{ \pi_1 }\left[ \left\vert \theta  \right\vert  \right]$. In the preceeding part we have proved that with the constraint satisfied with proper $ d $, we have $ \mathfrak{M}^* \gtrsim \varepsilon ^2 $.

Now solve the optimization problem by writing down the Lagrangian:
\begin{align*}
    \mathcal{L}(\pi_0,\pi_1, \lambda _1,\ldots,\lambda _d) =& \mathbb{E}_{ \pi_0 }\left[ \left\vert \theta  \right\vert  \right] -\mathbb{E}_{ \pi_1 }\left[ \left\vert \theta  \right\vert  \right] -\sum_{k=0}^d\lambda _k\big( \mathbb{E}_{ \pi_0 }\left[ \theta ^k \right] -\mathbb{E}_{ \pi_1 }\left[ \theta ^k \right]  \big)\\
    =& \mathbb{E}_{ \pi_0 }\left[ \left\vert \theta  \right\vert  -\sum_{k=0}^d\lambda _k\theta ^k  \right] -\mathbb{E}_{ \pi_1 }\left[ \left\vert \theta  \right\vert  -\sum_{k=0}^d\lambda _k\theta ^k  \right]
\end{align*}
optimize w.r.t. $ \pi_0,\pi_1 $ gives us:
\begin{align*}
     \mathop{ \max }\limits_{ \pi_0,\pi_1}\mathcal{L}=& \mathop{ \max  }\limits_{\theta \in [-1,1] }[\left\vert \theta  \right\vert - \mathrm{ poly }_d(\theta )] - \mathop{ \min }\limits_{\theta \in [-1,1]}[\left\vert \theta  \right\vert - \mathrm{ poly }_d(\theta )]
\end{align*}
then optimize w.r.t. $ \lambda _k $ suffices to choosing the optimal degree-$ d $ polynomial approximation of $ \left\vert x \right\vert  $, for which we have study in \autoref{sec:polyapprox}:
\begin{align*}
    \mathop{ \max }\limits_{\pi_0,\pi_1} \mathbb{E}_{ \pi_0 }\left[ \left\vert \theta  \right\vert  \right] -\mathbb{E}_{ \pi_1 }\left[ \left\vert \theta  \right\vert  \right]=&\min_{\lambda _{k=0}^d}\mathop{ \max }\limits_{ \pi_0,\pi_1}\mathcal{L} \\
    =& \mathop{ \inf }\limits_{G_d\in \mathcal{P}_d} \mathop{ \max }\limits_{\theta \in [-1,1]} \big\vert \left\vert \theta  \right\vert - G_d(\theta ) \big\vert \\
    =&\delta _d(\left\vert \, \cdot \,  \right\vert ) \\
    \asymp& \dfrac{ 1 }{ d }
\end{align*}


\begin{point}
    Result
\end{point}

Together we have:
\begin{align*}
    \mathfrak{M}^* \geq \big( p\mathop{ \max }\limits_{\pi_0,\pi_1} \mathbb{E}_{ \pi_0 }\left[ \left\vert \theta  \right\vert  \right] -\mathbb{E}_{ \pi_1 }\left[ \left\vert \theta  \right\vert  \right] \big)^2\big(1- d_\mathrm{ TV }\left(\mathbb{E}_{ \theta \sim \pi_1^{\otimes p} }\left[ p_\theta ^{\otimes p} \right] \big\Vert \mathbb{E}_{ \theta \sim \pi_0^{\otimes p} }\left[ p_\theta ^{\otimes p} \right] \right)  \big)
    \gtrsim (\dfrac{ p }{ d } )^2\Theta (1) 
\end{align*}
with the optimal $ d_p\asymp \dfrac{ \log p }{ \log\log p } $, we have:
\begin{align*}
    \mathfrak{M}^* \gtrsim p^2\left( \dfrac{ \log\log p }{ \log p } \right) ^2
\end{align*}



\subsection{Conclusion for Bounded Case}

Now putting back the scaling $ M $ we have:
\begin{align*}
    \mathfrak{M}^* \asymp M^2p^2\left( \dfrac{ \log\log p }{ \log p } \right) ^2 
\end{align*}

In the original paper \cite{cai2011testing} they provided the sharp Minimax risk:
\begin{align*}
    \mathfrak{M}^*(\Theta _p(M)) = \beta _*^2M^2p^2\left( \dfrac{ \log\log p }{ \log p } \right) ^2(1+o(1)),
\end{align*}
where $ \beta _* $ is the Berstein's constant  mentioned in \autoref{sec:polyapprox}.






\section{Upgrade to Unbounded Case}

Here we focus more on the spirit of how to upgrade the bounded case to unbounded case, and we will not go through the details of the proof.

The idea is that: For small $ \theta  $, we apply the preceeding polynomial approximation estimator, while for large $ \theta  $, we can safely use $ \left\vert X \right\vert  $ as an estimator of $ \left\vert \theta  \right\vert  $. The cut-off between the two case is estimated by the maximization of normal variable: For $ X_1,\ldots,X_p \mathop{ \sim }\limits^{i.i.d.}\mathcal{N}(0,1)  $, w.h.p.
\begin{align*}
     \max_{i=1}^p\left\vert X_i \right\vert  \leq 2\sqrt{\log p }.
\end{align*}

\subsection{Upper Bound Side}

For the upper bound side, we modify the "good enough" estimator to something look like $ \hat{T}= \mathrm{poly}_{d_p}(X)\mathbf{1}_{\left\vert X \right\vert \leq 2\sqrt{\log p}} + \left\vert X \right\vert \mathbf{1}_{\left\vert X \right\vert > 2\sqrt{\log p}}  $\footnote{To be specific, we split the data into pairs $ X=\{(X_{i1},X_{i2})\}_{i=1}^{n/2} $. In each pair, accroding to $ \left\vert X_{i1} \right\vert \gtrless  2\sqrt{\log p} $, we apply $ \mathrm{ poly }(X_{i2})  $ or $ \left\vert X_{i2} \right\vert  $ in the mean value $ \hat{T}=\sum_{i=1}^{n/2}\mathrm{ func }(X_{i2})  $. }. 


% Detail computation of bias and variance is omitted here. The final result is:
% \begin{align*}
%     \text{Bias: }&\left\vert \hat{T}(X)- \left\vert X \right\vert _1  \right\vert \lesssim \sqrt{\log p}\dfrac{ p }{ d }\\
%     \text{Variance: }&var(\hat{T}(X))\lesssim \sqrt{p}\log p
% \end{align*}
% which gives optimal $ d_p \sim \log p$
% \begin{align*}
%     \mathfrak{M}^* \lesssim 
% \end{align*}























% add bibtex file
\bibliographystyle{plain}
\bibliography{report}





\end{document}